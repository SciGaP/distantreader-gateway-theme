
@INPROCEEDINGS{Levi_Kyle_Rynge_Mats_Abeysinghe_Eroma_and_Edwards_Robert_A_undated-bg,
  title     = "Searching the Sequence Read Archive using Jetstream and Wrangler",
  booktitle = "{PEARC} 18",
  author    = "{Levi, Kyle Rynge, Mats Abeysinghe, Eroma, and Edwards, Robert
               A}"
}

@ARTICLE{Torres2017-zo,
  title     = "{PARTIE}: a partition engine to separate metagenomic and
               amplicon projects in the Sequence Read Archive",
  author    = "Torres, Pedro J and Edwards, Robert A and McNair, Katelyn A",
  abstract  = "Motivation: The Sequence Read Archive (SRA) contains raw data
               from many different types of sequence projects. As of 2017, the
               SRA contained approximately ten petabases of DNA sequence (10 16
               bp). Annotations of the data are provided by the submitter, and
               mining the data in the SRA is complicated by both the amount of
               data and the detail within those annotations. Here, we introduce
               PARTIE, a partition engine optimized to differentiate sequence
               read data into metagenomic (random) and amplicon (targeted)
               sequence data sets. Results: PARTIE subsamples reads from the
               sequencing file and calculates four different statistics: k -mer
               frequency, 16S abundance, prokaryotic- and viral-read abundance.
               These metrics are used to create a RandomForest decision tree to
               classify the sequencing data, and PARTIE provides mechanisms for
               both supervised and unsupervised classification. We demonstrate
               the accuracy of PARTIE for classifying SRA data, discuss the
               probable error rates in the SRA annotations and introduce a
               resource assessing SRA data. Availability and Implementation:
               PARTIE and reclassified metagenome SRA entries are available
               from https://github.com/linsalrob/partie. Contact:
               redwards@mail.sdsu.edu. Supplementary information: Supplementary
               data are available at Bioinformatics online.",
  journal   = "Bioinformatics",
  publisher = "Oxford University Press",
  volume    =  33,
  number    =  15,
  pages     = "2389--2391",
  month     =  aug,
  year      =  2017,
  language  = "en"
}

@ARTICLE{Towns2014-po,
  title    = "{XSEDE}: Accelerating Scientific Discovery",
  author   = "Towns, J and Cockerill, T and Dahan, M and Foster, I and Gaither,
              K and Grimshaw, A and Hazlewood, V and Lathrop, S and Lifka, D
              and Peterson, G D and Roskies, R and Scott, J R and
              Wilkins-Diehr, N",
  abstract = "Computing in science and engineering is now ubiquitous: digital
              technologies underpin, accelerate, and enable new, even
              transformational, research in all domains. Access to an array of
              integrated and well-supported high-end digital services is
              critical for the advancement of knowledge. Driven by community
              needs, the Extreme Science and Engineering Discovery Environment
              (XSEDE) project substantially enhances the productivity of a
              growing community of scholars, researchers, and engineers
              (collectively referred to as ``scientists''' throughout this
              article) through access to advanced digital services that support
              open research. XSEDE's integrated, comprehensive suite of
              advanced digital services federates with other high-end
              facilities and with campus-based resources, serving as the
              foundation for a national e-science infrastructure ecosystem.
              XSEDE's e-science infrastructure has tremendous potential for
              enabling new advancements in research and education. XSEDE's
              vision is a world of digitally enabled scholars, researchers, and
              engineers participating in multidisciplinary collaborations to
              tackle society's grand challenges.",
  journal  = "Computing in Science Engineering",
  volume   =  16,
  number   =  5,
  pages    = "62--74",
  year     =  2014,
  keywords = "engineering computing;natural sciences computing;research and
              development management;research initiatives;Extreme Science and
              Engineering Discovery Environment;XSEDE;advanced digital
              services;campus-based resources;digital services;digitally
              enabled scholars;multidisciplinary collaborations;national
              e-science infrastructure ecosystem;scientific discovery;Digital
              systems;Knowledge discovery;Materials engineering;Scientific
              computing;Supercomputers;HPC;cyberinfrastructure;distributed
              computing;distributed virtual organizations;research
              infrastructures;scientific computing"
}

@INPROCEEDINGS{Stewart2015-pr,
  title     = "Jetstream: a self-provisioned, scalable science and engineering
               cloud environment",
  booktitle = "Proceedings of the 2015 {XSEDE} Conference: Scientific
               Advancements Enabled by Enhanced Cyberinfrastructure",
  author    = "Stewart, Craig A and Cockerill, Timothy M and Foster, Ian and
               Hancock, David and Merchant, Nirav and Skidmore, Edwin and
               Stanzione, Daniel and Taylor, James and Tuecke, Steven and
               Turner, George and Vaughn, Matthew and Gaffney, Niall I",
  publisher = "ACM",
  pages     = "29",
  month     =  jul,
  year      =  2015,
  keywords  = "atmosphere; big data; cloud computing; long tail of science"
}

@ARTICLE{Buchfink2015-nz,
  title    = "Fast and sensitive protein alignment using {DIAMOND}",
  author   = "Buchfink, Benjamin and Xie, Chao and Huson, Daniel H",
  abstract = "The alignment of sequencing reads against a protein reference
              database is a major computational bottleneck in metagenomics and
              data-intensive evolutionary projects. Although recent tools offer
              improved performance over the gold standard BLASTX, they exhibit
              only a modest speedup or low sensitivity. We introduce DIAMOND,
              an open-source algorithm based on double indexing that is 20,000
              times faster than BLASTX on short reads and has a similar degree
              of sensitivity.",
  journal  = "Nat. Methods",
  volume   =  12,
  number   =  1,
  pages    = "59--60",
  month    =  jan,
  year     =  2015,
  language = "en"
}

@ARTICLE{Langmead2012-gu,
  title    = "Fast gapped-read alignment with Bowtie 2",
  author   = "Langmead, Ben and Salzberg, Steven L",
  abstract = "As the rate of sequencing increases, greater throughput is
              demanded from read aligners. The full-text minute index is often
              used to make alignment very fast and memory-efficient, but the
              approach is ill-suited to finding longer, gapped alignments.
              Bowtie 2 combines the strengths of the full-text minute index
              with the flexibility and speed of hardware-accelerated dynamic
              programming algorithms to achieve a combination of high speed,
              sensitivity and accuracy.",
  journal  = "Nat. Methods",
  volume   =  9,
  number   =  4,
  pages    = "357--359",
  month    =  apr,
  year     =  2012,
  language = "eng"
}



